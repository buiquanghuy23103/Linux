\chapter{Text files}

\section{Redirecting output}

\tableStart[\caption{Channels (File descriptor)}] { |p{2\xm}|p{3\xm}|p{3\xm}|p{2\xm}|p{3\xm}| }
\head{Number} & \head{Channel} & \head{Description} & \head{Default connection} & \head{Usage}\w
0 & \verb|stdin| & Standard input & keyboard & read only\w
1 & \verb|stdout| & Standard output & terminal & write only\w
2 & \verb|stderr| & Standard error & terminal & write only\w
3+ & \verb|filename| & Other files & none & read and/or write only\w
\tableEnd

\tableStart[\caption{Output redirection operators}] {|l|l|}
\head{Usage} & \head{Explanation}\w
\verb|>file| & redirect \verb|stdout| to overwrite the file\w
\verb|>>file| & redirect \verb|stdout| to append to the file\w
\verb|2>file| & redirect \verb|stderr| to append to the file\w
\verb|2>/dev/null| & discard error messages by redirecting to \verb|/dev/null|\w
\verb|&>file| & redirect \verb|stdout| and \verb|stderr| to overwrite the same file\w
\verb|&>>file| & redirect \verb|stdout| and \verb|stderr| to append to the same file\w
\tableEnd

The order of redirection is important. The sequence \verb|> file 2>&1| redirects redirect \verb|stdout| to a file (\verb|>file|), then redirect \verb|stderr| (\verb|2>|) to the same place that \verb|stdout| is directed to (\verb|&1|, 1 is the number of \verb|stdout|). However, the sequence \verb|2>&1 > file| redirects \verb|stderr| (\verb|2>|) to the place of \verb|stdout| (\verb|&1|), meaning the terminal window. It then redirects only the \verb|stdout| to a file.

\begin{itemize}
\item Standard error can be redirected through a pipe, but the merging operator such as \code{&>} or \code{&>>} cannot be used. The following command is the correct way to redirect both \verb|stdout| and \verb|stderr| through a pipe.
\begin{commandshell}
find /etc -name passwd 2>&1 | less
\end{commandshell}

\item The \code{tee} command redirects \verb|stdout| to the terminal window, and at the same time, passes it to some other program through a pipe. The following command prints the output of \code{ls -l} command on the terminal window as well as redirects \verb|stdout| to a file.
\begin{commandshell}
ls -l | tee /tmp/output
\end{commandshell}

\item In the following example,  \verb|/dev/pts/0| is the device file of that represents the current terminal window. The command prints the \verb|stdout| of \code{ls -l} command on the terminal window as well as sends it to a specified email.
\begin{commandshell}
ls -l | tee /dev/pts/0 | mail huy.bui@edu.xamk.fi
\end{commandshell}

\item Save the output to a file and discard error messages
\begin{commandshell}
find /etc -name passwd > /tmp/output 2> /dev/null
\end{commandshell}

\item Store output and error messages to the same file
\begin{commandshell}
find /etc -name passwd &> /tmp/output-errors
\end{commandshell}

\item Append output and error messages to an existing file
\begin{commandshell}
find /etc -name passwd >> /tmp/output-errors 2>&1
\end{commandshell}

\item Copy 10 lines from a log file and append them to another file
\begin{commandshell}
tail -n 10 /var/lo/dmseg >> /tmp/boot-messages
\end{commandshell}

\item Concentrate four files into one
\begin{commandshell}
cat file1 file2 file3 file4 > /tmp/four-in-one
\end{commandshell}

\item Save output and error messages to separate files
\begin{commandshell}
find /etc -name passwd > /tmp/output 2> /tmp/error
\end{commandshell}

\end{itemize}

\section{Edit text files}