\chapter{Text files}

\section{Redirecting output}

\tableStart[\caption{Channels (File descriptor)}] { |p{2\xm}|p{3\xm}|p{3\xm}|p{2\xm}|p{3\xm}| }
\head{Number} & \head{Channel} & \head{Description} & \head{Default connection} & \head{Usage}\w
0 & \verb|stdin| & Standard input & keyboard & read only\w
1 & \verb|stdout| & Standard output & terminal & write only\w
2 & \verb|stderr| & Standard error & terminal & write only\w
3+ & \verb|filename| & Other files & none & read and/or write only\w
\tableEnd

\tableStart[\caption{Output redirection operators}] {|l|l|}
\head{Usage} & \head{Explanation}\w
\verb|>file| & redirect \verb|stdout| to overwrite the file\w
\verb|>>file| & redirect \verb|stdout| to append to the file\w
\verb|2>file| & redirect \verb|stderr| to append to the file\w
\verb|2>/dev/null| & discard error messages by redirecting to \verb|/dev/null|\w
\verb|&>file| & redirect \verb|stdout| and \verb|stderr| to overwrite the same file\w
\verb|&>>file| & redirect \verb|stdout| and \verb|stderr| to append to the same file\w
\tableEnd

The order of redirection is important. The sequence \verb|> file 2>&1| redirects redirect \verb|stdout| to a file (\verb|>file|), then redirect \verb|stderr| (\verb|2>|) to the same place that \verb|stdout| is directed to (\verb|&1|, 1 is the number of \verb|stdout|). However, the sequence \verb|2>&1 > file| redirects \verb|stderr| (\verb|2>|) to the place of \verb|stdout| (\verb|&1|), meaning the terminal window. It then redirects only the \verb|stdout| to a file.

\begin{itemize}
\item Standard error can be redirected through a pipe, but the merging operator (\verb|&>| or \verb|&>>|) cannot be used. The following command is the correct way to redirect both \verb|stdout| and \verb|stderr| through a pipe.
\begin{commandshell}
find /etc -name passwd 2>&1 | less
\end{commandshell}

\item The \code{tee} command redirects \verb|stdout| to the terminal window, and at the same time, passes it to some other program through a pipe. The following command prints the output of \code{ls -l} command on the terminal window as well as redirects \verb|stdout| to a file.
\begin{commandshell}
ls -l | tee /tmp/output
\end{commandshell}

\item In the following example,  \verb|/dev/pts/0| is the device file of that represents the current terminal window. The command prints the \verb|stdout| of \code{ls -l} command on the terminal window as well as sends it to a specified email.
\begin{commandshell}
ls -l | tee /dev/pts/0 | mail huy.bui@edu.xamk.fi
\end{commandshell}

\item Save the output to a file and discard error messages
\begin{commandshell}
find /etc -name passwd > /tmp/output 2> /dev/null
\end{commandshell}

\item Store output and error messages to the same file
\begin{commandshell}
find /etc -name passwd &> /tmp/output-errors
\end{commandshell}

\item Append output and error messages to an existing file
\begin{commandshell}
find /etc -name passwd >> /tmp/output-errors 2>&1
\end{commandshell}

\item Copy 10 lines from a log file and append them to another file
\begin{commandshell}
tail -n 10 /var/lo/dmseg >> /tmp/boot-messages
\end{commandshell}

\item Concentrate four files into one
\begin{commandshell}
cat file1 file2 file3 file4 > /tmp/four-in-one
\end{commandshell}

\item Save output and error messages to separate files
\begin{commandshell}
find /etc -name passwd > /tmp/output 2> /tmp/error
\end{commandshell}

\end{itemize}

\section{Vim editor}

Vim is a highly configurable and efficient editor for practiced users, including such features as split screen editing, color formatting, and highlighting command syntax. When first opened, Vim starts in \emph{command mode}, used for text manipulation (navigation, cut, paste, delete, etc.). Enter each of other modes with a single character keystrokes:

\begin{itemize}
\item An \code{i} keystroke enters \emph{insert mode}, where all text typed becomes file content. Press \code{Esc} to return to command mode.

\item A \code{v} keystroke enters \emph{visual mode}, where multiple characters may be selected for text manipulation. Use \code{V} for multi-line selection and \code{Ctrl+v} for block selection. The same keystroke used to enter the visual mode (\code{v}, \code{Ctrl+v}, \code{V}) is used to exit.

\item The \code{:} keystroke begins \emph{extended command mode} for tasks like saving file, exiting Vim editor.
\end{itemize}

The following workflow presents the minimum keystrokes every Vim user must learn:

\begin{enumerate}
\item Open a file with \code{vim <filename>}
\item Use arrow keys to move the cursor
\item Edit file content in insert mode
    \begin{itemize}
    \item Press \code{i} to enter insert mode, then edit the file content
    \item Press \code{Esc} to return to command mode
    \end{itemize}
\item Text manipulation in command mode
    \begin{itemize}
    \item Press \code{u} to undo mistaken edits on the current line
    \item Press \code{x} to delete a selection of text
    \end{itemize}      
\item Rearranging existing text
    \begin{itemize}
    \item In command mode, press \code{v} to enter visual mode. Use arrow keys to position the cursor to the last character
    \item Press \code{y} to yank (copy) the selection. Move the cursor to position the insert location. Press \code{p} to paste the selection
    \end{itemize}        
\item Save files or exit
    \begin{itemize}
    \item Enter \code{:w} to write (save) the file and remain in command mode
    \item Enter \code{:wq} to write the file and quit Vim
    \item Enter \code{:q!} to quit Vim but don't save file
    \end{itemize}        
\end{enumerate}