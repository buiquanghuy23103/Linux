\chapter{Permissions and Ownerships}

\section{Permission}

\subsection{Introduction}

Permissions are identified with either the letter abbreviation (\verb|r|, \verb|w|, \verb|x|, etc.), or the octal value that corresponds to the permission. In letter abbreviation: if a given permission has not been assigned, a dash (-) takes its place in the mode. 

\begin{verbatim}
[user1@host1 ~]$ ls -l
-rw-rw-r--. 1 user1 user1 0 Feb 8 17:36 test.txt
[user1@host1 ~]$ ls -ld /home
drw-rw-r--. 1 user1 user1 0 Feb 8 17:36 /home
\end{verbatim}

\tableStart[Effects of permissions on files and directories] {|l|p{3\xm}|p{7\xm}| }
\head{Permission} & \head{Effect on files} & \head{Effect on directories}\w
\verb|r| (read) & Contents of the file can be read & Content of the directory (list of file names) can be shown\w
\verb|w| (write) & Contents of the file can be changed & Any file in the directory can be created or deleted\w
\verb|x| (execute) & Files can be executed as commands & Contents of the directory can be accessed\w
\tableEnd

A user normally has to have both \verb|read| and \verb|execute| on a read-only directory, so that the content of the directory can be listed. If a user only has \verb|read| permission on a directory, then except for the list of all files, no other information such as permissions, time stamps, can be accessed. If users only have \verb|execute| permission on a directory, they cannot see the list of all file names in the directory, but they can access any file in the directory as long as they know the file names.\\

The \verb|ls -l| command will expand the file listing including file's permissions and ownerships. The \verb|ls -ld| command does the same for directories.

\subsection{Configuration}

The command \verb|chmod| can change permissions of files and directories. The permission instruction can be issued in either symbolic form or numeric form.\\

\begin{verbatim}
chmod WhoWhatWhich file|directory
\end{verbatim}

The symbolic method of change permissions uses letters to represent different groups of permissions: \verb|u| for user, \verb|g| for group, \verb|other| for other, and \verb|a| for all. In order to change permission, use three symbols: \verb|+| to add permissions, \verb|-| to remove permissions, and \verb|=| to replace the entire set of permissions. For example, the following command removes write and read permission from group and other on \verb|file.txt|

\begin{commandshell}
chmod go-rw file.txt
\end{commandshell}

The  following command execute permission to everyone on \verb|file.sh|

\begin{commandshell}
chmod a+x file.sh
\end{commandshell}

As for numeric method of changing permissions, each digit \verb|#| represents an access level: user, group, and other. Each digit \verb|#| is the sum of \verb|r|=4, \verb|w|=2, and \verb|x|=1. Examine the permission \verb|-rwxr-x---|. For the user, \verb|rwx| is calculated as 4+2+1=7. For the group \verb|r-x| is calculated as 4+0+1=5, for other is calculated as 0+0+0=0. Putting these three together, the numeric representation is 750.

\begin{verbatim}
chmod ### file|directory
\end{verbatim}

The following command set permissions for \verb|sampledir| file:

\begin{commandshell}
chmod 750 sampledir
\end{commandshell}

\section{Ownership}

\subsection{Basic configuration}

When a file is created, its owner is the user who created it, and the owning group is the user's current group. The \verb|chown| command can change these values to something else. The syntax of this command is shown as below:

\begin{verbatim}
chown <option> <ownership> <file>
\end{verbatim}

Specifically, there are five ways to format \verb|<ownership|:

\begin{itemize}
\item \verb|user.group| or \verb|user:group| -- The user and group to own the file, separated by a colon (or dot), with no spaces in between.
\item \verb|user| -- The name of the user to own the file. In this form, the colon (":") and the group is omitted. The owning group is not altered.
\item \verb|.group| or \verb|:group| -- The group to own the file. In this form, user is omitted, and the group must be preceded by a colon.
\item \verb|user:| or \verb|user.| -- If group is omitted, but a colon follows user, the owner is changed to user, and the owning group is changed to the login group of user.
\end{itemize}

For example, the following command would grant ownership of \verb|file.txt| to user \verb|visitor| and group \verb|guests|.

\begin{commandshell}
chown visitor:guests file.txt
\end{commandshell}

\verb|chown| can be used with the \verb|-R| option to recursively change the ownership of the entire directory tree. The following command would grant ownership of \verb|foodir| along with all files and subdirectories within it to user \verb|student|.

\begin{commandshell}
chown -R student ~/foodir/
\end{commandshell}

