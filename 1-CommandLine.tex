\chapter{The Shell}

\section{Basic commands and tips}

\tableStart[\caption{Shell shortcuts}] {|l|l|}
\head{Shortcuts} & \head{Explanation} \w
\code{Ctrl+Alt+F1-F6} & Start or switch to the first (to sixth) session \w
Ctrl + Alt + F7 & Switch to GUI session \w
Shift + PgUp & Scroll up\w 
Shift + PgDown & Scroll down\w 
Ctrl + A & Jump to the beginning of the line \w
Ctrl + E & Jump to the end of the line \w
Ctrl + U & Clear from the cursor to the beginning of the line \w
Ctrl + K & Clear from the cursor to the end of the line \w
!string & Re-execute a recent command by matching the command name \w
!65 & Re-execute a 65th command in the history list \w
Ctrl + R & Search the history list of commands for a pattern \w
Esc + . & Copy the last argument of the previous command to the current one \w
Ctrl + C & Stop the running command \w
Ctrl + Z & Suspend the running command \w
\tableEnd


\tableStart[\caption{Basic Files and Directory commands}] {|l|l| p{9cm} |}
\head{Activity} & \head{Command} & \head{Explain} \w

Copy a file & \verb|cp file1 file2| & Copy \verb|file1| to a desire location and change name to \verb|file2| \w

Copy multiple files & \verb|cp file1 file2 file3 dir| & Copy \verb|file1|, \verb|file2|, and \verb|file3| to directory \verb|dir|. Note that the last argument must be a directory. \w

Move a file & \verb|mv file1 file2| & If \verb|file2| resides in a different directory than \verb|file1|, then \verb|file1| is moved to the new location \w

Rename a file & \verb|mv file1 file2| & If \verb|file2| and \verb|file1| are in the same directory, then \verb|file1| is renamed as \verb|file2| \w

Move multiple files & \verb|mv file1 file2 file3 dir| & Move \verb|file1|, \verb|file2|, and \verb|file3| to directory \verb|dir|\w

Create directory & \verb|mkdir dir| & Create directory \verb|dir| under the current directory\w

Create directory path & \verb|mkdir -p ~/dir1/dir2/dir3| & Create \verb|dir1|,\verb|dir2|, and \verb|dir3| to match the specified directory structure \verb|~/dir1/dir2/dir3| \w

Copy directories & \verb|cp -r dir1 dir2 dir3 dir4| & Recursively copy \verb|dir1|,\verb|dir2|, and \verb|dir3| to \verb|dir4| \w

Move directories & \verb|mv dir1 dir2 dir3 dir4| & Move \verb|dir1|,\verb|dir2|, and \verb|dir3| to \verb|dir4| \w

Remove directories & \verb|rm -rf dir1 dir2 dir3| & Remove (without questioning) \verb|dir1|,\verb|dir2|, and \verb|dir3| \w
\tableEnd

\tableStart[\caption{Common meta-characters and pattern classes}] {|l|l|}
\head{Characters}&\head{Explanation}\w
\verb|*| & Any string of zero or more characters\w
\verb|?| & Any single character\w
\verb|~| & The current user's home directory\w
\verb|~kevin| & The current Kevin's home directory\w
\verb|~+| & The current working directory\w
\verb|~-| & The previous working directory\w
\verb|[abc...]| & Any one character in the enclosed class\w
\verb|[!abc...]| & Any one character \emph{not} in the enclosed class\w
\verb|[^abc...]| & Any one character \emph{not} in the enclosed class\w
\verb|[[:alpha:]]| & Any alphabetic character\w
\verb|[[:alnum:]]| & Any alphabetic character or digit\w
\verb|[[:lower:]]| & Any lower-case character\w 
\verb|[[:upper:]]| & Any upper-case character\w
\verb|[[:punct:]]| & Printable character (not a space or alphanumeric)\w  
\verb|[[:digit:]]| & Any digit, 0 -- 9\w 
\verb|[[:space:]]| & Any one whitespace character (space, tab, newline, carriage return, form feeds)\w
\verb|{a,b,2,3}| & Any characters in the brace \w  
\verb|{1..8}| & Any single digit from 1 to 8\w
\verb|{a..d}| & Any single characters from a to d (a, b, c, d) \w
\tableEnd

\paragraph{Command substitution:} allows output of a command to replace itself. It occurs when a command is enclosed with a dollar sign and a pair of parentheses \verb|$(command)|, or backticks \verb|`command`|. The form with backticks cannot be nested inside another pair of backticks. In contrast, \verb|$(command)| can nest multiple expansions inside each other.

\begin{verbatim}

\end{verbatim}

\begin{commandshell}
echo Today is `date`
cat $(fgrep -l ens32 /var/*)
echo Today is $(date)
\end{commandshell}

\paragraph{Protecting argument from expansion:} To ignore meta-character special meanings, we use \emph{escaping} and \emph{quoting}. The backslash \verb|\| is an escape character that protects the single following character from special interpretation. The single quotes \verb|' '| and double quotes \verb|"  "| are used to enclose a string. The double quotes suppress shell expansions and meta-characters but still allows command and variable substitutions. The single quotes interpret all text literally, it suppress shell expansions and meta-characters as well as command and variable substitutions. 

\begin{verbatim}
kevin$> echo "**hostname is $(host)**"
**hostname is rtracy**

kevin$> echo '**hostname is $(host)**'
**hostname is $(host)**
\end{verbatim}

\section{Configuration files}

Shell configuration files are scripts that execute when a shell starts. There are two types of shells:

\begin{itemize}
\item Login shell: run when the system starts and is using only the CLI as the user interface.
\item Non-login shell run when a shell session is opened from within the GUI. 
\end{itemize}

Login shells execute only one of the following configuration files in order (means that if a file is found, all files followed it are abandoned):

\begin{enumerate}
\item \verb|/etc/profile|
\item \verb|~/.bash_profile|
\item \verb|~/.bash_login|
\item \verb|~/.profile|
\end{enumerate}

\tableStart[\caption{Shell configuration files}] {|l|l| p{10cm} |}
\head{File} & \head{Type} & \head{Explanation} \w
\verb|/etc/profile.d/| & login & Configurations in this directory are applied for all users. Do not modify the system's files, instead create your own file (for example: \verb|MyConf.sh|) and add custom configurations  to it.\w
\verb|~/.bash_profile| & login & Contains shell preferences for individual user. \w
\verb|~/.profile| & login & Contains environment variables for individual users. \w
\verb|~./bash_login| & login & Store commands executed when a user logs in. \w
\verb|~./bash_logout| & login & Store commands executed when a user logs out. \w
\verb|/etc/bashrc| & non-login & Contain shell preferences for all users. \w
\verb|~/.bashrc| & non-login & Pass variables between shells. \w
\tableEnd